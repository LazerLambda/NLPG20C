For this task, we used the NLTK library to calculate the number of unique tokens that exist in a review. Then, we plotted the graphs to show the distributions of the tokens, both with and without stemming.

We can see that Figure 2 shows the number of unique tokens without stemming in every review, while Figure 3 shows the number of unique tokens after stemming is performed in every review.

To make the comparison easier, in Figure 4, the stems are indicated with the color orange, while the tokens are indicated with the color blue.

Based on Figure 4, we can conclude that the number of unique tokens with stemming is generally lower that the number of unique tokens without performing stemming. This is because some different tokens may have the same stems.

We also listed the 20 most frequent tokens in all reviews before and after performing stemming. Based on the dataset type, which is a business review, we would expect the reviews to have the words "place", "service", "great", "bad", and other similar words to describe the business.

Based on Table 2 and Table 3, the 3 most common words that appear are "food", "place", and "good", which are the words that we would expect to have on the reviews.