After reviewing a few reviews on Business 1, it became obvious, that this business is a car wash. The reviews were mixed, some criticized alot and other reviews were mentioning the good result. \\
Business 2 was also a non restaurant business. The noun-adjective pairs such as (‘front’,’desk’), (‘free’,’wifi’), (‘rental’, ‘car’), (‘complimentary’, ‘breakfast’) suggest that the reviews are about an hotel which is true. We also get a general idea from the pairs that the staff, breakfast as well as the facilities are decent which is also true from reading the reviews. Apart from some complaints here and there in the reviews, the guests generally liked the hotel and its free shuttle service. However, some pairs like (‘light’, ‘sleeper’) is useless as it talks about the guest. Also, (‘next’, ‘day’) and (‘first’, ‘day’) occur frequently which does not give us any info.\\
As for Business 3, most reviews spoke great of the dumplings as well as the food, which are the 2 most frequent noun-adjective pairs. Hence, those are accurate. As from the reviews, it is given that the restaurant has a mix of Korean as well as Northern-Chinese dishes which is also something that can be rightly perceived from the frequent noun-adjective pairs obtained, e.g, (‘korean’, ‘food’), (‘korean’, ‘dishes’), (‘chinese’, ‘cuisine’), etc. Fried pork and fried rice are some of the dishes that are sold in the restaurant that people talked of in the reviews. Most reviews talk of how bad the service is. Apparently, the management and staff is not friendly, and the food is brought in quite slowly. This is in contradiction with one of the noun-adjective pairs that we got ((‘great’), (‘service’)).  Perhaps, the noun-adjective pair could not comprehend the meaning of the sentences where service and great would appear together. These are some sentences quoted from the reviews: “If you are expecting great service, this the wrong place for you”, “but service is not so great”.  Therefore, when extracting noun-adjective pairs for these sentences, we obtain (‘great’), (‘service’), although the expression of the sentence is actually negative.\\
Business 4 represents a South Indian restaurant with a buffet set-up. Hence, the first three most frequent adj-noun pairs correctly summarizes the nature of the restaurant and the kind of food being served in the restaurant. Noun-adj pairs like (‘first’, ‘time’) do not hold much importance in the feedback of the restaurant but (‘second’, ‘time’) can refer to customers wanting to come back which speaks good of the restaurant. Besides, this is quite true because there are quite a few reviews that talks about customers wanting to give it another try. Most reviews also speak good of the staff as given by the (‘friendly’, ‘staff’) pair. Most spoken of food item is the dosa and that’s also given as one of the noun-adjective pairs. On the whole, the frequent noun-adjective pairs obtained give an accurate response of the restaurant.\\
For business 5 we found 4 good reviews and 1 bad review. It seems like busines 5 is a mexican restaurant, which is also indicated by the ('mexican', 'restaurants'), 3)- Pair. We also found the pair ('carne', 'asada'), 13) as "carne asada" in the reviews again.\\