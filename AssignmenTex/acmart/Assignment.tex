 
%%
%% This is file `sample-sigchi.tex',
%% generated with the docstrip utility.
%%
%% The original source files were:
%%
%% samples.dtx  (with options: `sigchi')
%% 
%% IMPORTANT NOTICE:
%% 
%% For the copyright see the source file.
%% 
%% Any modified versions of this file must be renamed
%% with new filenames distinct from sample-sigchi.tex.
%% 
%% For distribution of the original source see the terms
%% for copying and modification in the file samples.dtx.
%% 
%% This generated file may be distributed as long as the
%% original source files, as listed above, are part of the
%% same distribution. (The sources need not necessarily be
%% in the same archive or directory.)
%%
%% The first command in your LaTeX source must be the \documentclass command.
\documentclass[sigchi]{acmart}

\usepackage{spverbatim}
\usepackage{graphicx}

%%
%% \BibTeX command to typeset BibTeX logo in the docs
\AtBeginDocument{%
	\providecommand\BibTeX{{%
			\normalfont B\kern-0.5em{\scshape i\kern-0.25em b}\kern-0.8em\TeX}}}

%% Rights management information.  This information is sent to you
%% when you complete the rights form.  These commands have SAMPLE
%% values in them; it is your responsibility as an author to replace
%% the commands and values with those provided to you when you
%% complete the rights form.
\setcopyright{acmcopyright}
\copyrightyear{2018}
\acmYear{2018}
\acmDOI{10.1145/1122445.1122456}

%% These commands are for a PROCEEDINGS abstract or paper.
%\acmConference[Woodstock '18]{Woodstock '18: ACM Symposium on Neural%
%	Gaze Detection}{June 03--05, 2018}{Woodstock, NY}
%\acmBooktitle{Woodstock '18: ACM Symposium on Neural Gaze Detection,
%	June 03--05, 2018, Woodstock, NY}
%\acmPrice{15.00}
%\acmISBN{978-1-4503-9999-9/18/06}


%%
%% Submission ID.
%% Use this when submitting an article to a sponsored event. You'll
%% receive a unique submission ID from the organizers
%% of the event, and this ID should be used as the parameter to this command.
%%\acmSubmissionID{123-A56-BU3}

%%
%% The majority of ACM publications use numbered citations and
%% references.  The command \citestyle{authoryear} switches to the
%% "author year" style.
%%
%% If you are preparing content for an event
%% sponsored by ACM SIGGRAPH, you must use the "author year" style of
%% citations and references.
%% Uncommenting
%% the next command will enable that style.
%%\citestyle{acmauthoryear}

%%
%% end of the preamble, start of the body of the document source.
\begin{document}
	
	%%
	%% The "title" command has an optional parameter,
	%% allowing the author to define a "short title" to be used in page headers.
	\title{Natural Language Processing CZ4045}
	\subtitle{Group Report (G20C)}
	
	%%
	%% The "author" command and its associated commands are used to define
	%% the authors and their affiliations.
	%% Of note is the shared affiliation of the first two authors, and the
	%% "authornote" and "authornotemark" commands
	%% used to denote shared contribution to the research.
	\author{Ben Trovato}
	\authornote{Both authors contributed equally to this research.}
	\email{trovato@corporation.com}
	\orcid{1234-5678-9012}
	\author{G.K.M. Tobin}
	\authornotemark[1]
	\email{webmaster@marysville-ohio.com}
	\affiliation{%
		\institution{Institute for Clarity in Documentation}
		\streetaddress{P.O. Box 1212}
		\city{Dublin}
		\state{Ohio}
		\postcode{43017-6221}
	}
	
	\author{Lars Th{\o}rv{\"a}ld}
	\affiliation{%
		\institution{The Th{\o}rv{\"a}ld Group}
		\streetaddress{1 Th{\o}rv{\"a}ld Circle}
		\city{Hekla}
		\country{Iceland}}
	\email{larst@affiliation.org}
	
	\author{Valerie B\'eranger}
	\affiliation{%
		\institution{Inria Paris-Rocquencourt}
		\city{Rocquencourt}
		\country{France}
	}
	
	\author{Aparna Patel}
	\affiliation{%
		\institution{Rajiv Gandhi University}
		\streetaddress{Rono-Hills}
		\city{Doimukh}
		\state{Arunachal Pradesh}
		\country{India}}
	
	\author{Huifen Chan}
	\affiliation{%
		\institution{Tsinghua University}
		\streetaddress{30 Shuangqing Rd}
		\city{Haidian Qu}
		\state{Beijing Shi}
		\country{China}}
	
	\author{Charles Palmer}
	\affiliation{%
		\institution{Palmer Research Laboratories}
		\streetaddress{8600 Datapoint Drive}
		\city{San Antonio}
		\state{Texas}
		\postcode{78229}}
	\email{cpalmer@prl.com}
	
	\author{John Smith}
	\affiliation{\institution{The Th{\o}rv{\"a}ld Group}}
	\email{jsmith@affiliation.org}
	
	\author{Julius P. Kumquat}
	\affiliation{\institution{The Kumquat Consortium}}
	\email{jpkumquat@consortium.net}
	
	%%
	%% By default, the full list of authors will be used in the page
	%% headers. Often, this list is too long, and will overlap
	%% other information printed in the page headers. This command allows
	%% the author to define a more concise list
	%% of authors' names for this purpose.
	% \renewcommand{\shortauthors}{Trovato and Tobin, et al.}
	
	%%
	%% The abstract is a short summary of the work to be presented in the
	%% article.
	\begin{abstract}
		Our task covered data processing on a dataset provided by the review platform \textit{yelp}. We had to analyze the data
		descriptively and we had to focus on the Adjectives in the reports. Therefore we had to compare different methods on how the reviews can be represented by adjectives, which also became our application model. In our application model we were able to find specific properties of the business reviewed in the data.
	\end{abstract}
	
	%%
	%% The code below is generated by the tool at http://dl.acm.org/ccs.cfm.
	%% Please copy and paste the code instead of the example below.
	%%
	\begin{CCSXML}
		<ccs2012>
		<concept>
		<concept_id>10010520.10010553.10010562</concept_id>
		<concept_desc>Computer systems organization~Embedded systems</concept_desc>
		<concept_significance>500</concept_significance>
		</concept>
		</ccs2012>
	\end{CCSXML}
	
	\ccsdesc[500]{Natural Language Processing~Group Assignment}

	
	%%
	%% Keywords. The author(s) should pick words that accurately describe
	%% the work being presented. Separate the keywords with commas.
	\keywords{datasets, neural networks, gaze detection, text tagging}
	
	
	%%
	%% This command processes the author and affiliation and title
	%% information and builds the first part of the formatted document.
	\maketitle
	
	\section{Dataset Analysis}
	\subsection{Writing Style}
	For the sentence segmentation we used the library spacy. Each category is displayed in the graph below.
	\begin{center}
		\begin{table}[!h]
			\caption{Average length of the sentences in the different star categories}
			\begin{tabular}{ c c c c c}
				1 Star & 2 Star & 3 Star  & 4 Star & 5 Star\\
				30.5 & 25.9 & 24.0  & 25 & 24\\
			\end{tabular}
		\end{table}
	\end{center}
	
	\begin{figure}
		\caption{Histograms of the length of the sentences}
		\includegraphics[scale=0.3]{figures/1stars-Sentencelength.png}
		\includegraphics[scale=0.3]{figures/2stars-Sentencelength.png}
		\includegraphics[scale=0.3]{figures/3stars-Sentencelength.png}
		\includegraphics[scale=0.3]{figures/4stars-Sentencelength.png}
		\includegraphics[scale=0.3]{figures/5stars-Sentencelength.png}
	\end{figure}
	As can be seen, the length of sentences is on average longer with a poor rating than with a good rating. 
	
	\subsection{Sentence Segmentation}
	
	\subsection{Tokenization and Stemming}
	For this task, we used the NLTK library to calculate the number of unique tokens that exist in a review. Then, we plotted the graphs to show the distributions of the tokens, both with and without stemming.

We can see the figure below. Figure 2 shows the number of unique tokens without stemming in every review, while Figure 3 shows the number of unique tokens after stemming is performed in every review.

\begin{figure}
    \centering
    \caption{Distributions of tokens in each review (without stemming)}
    \includegraphics[scale=0.54]{figures/token_review.png}
    \label{fig:tokenized_review}
\end{figure}

\begin{figure}
    \centering
    \caption{Distributions of tokens in each review (with stemming)}
    \includegraphics[scale=0.5]{figures/stem_review.png}
    \label{fig:stemmed_token_review}
\end{figure}

To make the comparison easier, in Figure 4, the stems are indicated with the color orange, while the tokens are indicated with the color blue.

Based on Figure 4, we can conclude that the number of unique tokens with stemming is generally lower that the number of unique tokens without performing stemming.

\begin{figure}
    \centering
    \caption{Comparison of the distributions of the tokens with and without stemming}
    \includegraphics[scale=0.5]{figures/token_stem_review.png}
    \label{fig:stem_and_token_review}
\end{figure}

\begin{table}[]
	\tiny
    \centering
    \caption{20 most common unique tokens}
    \begin{tabular}{c|c|c}
        No. & Token & No. of Appearances  \\
        \hline
        1 & food & 8580 \\
        2 & place & 8236 \\
        3 & good & 7919 \\
        4 & great & 6295 \\
        5 & service & 6027 \\
        6 & like & 5534 \\
        7 & get & 5216 \\
        8 & time & 5172 \\
        9 & would & 5168 \\
        10 & one & 5044 \\
        11 & back & 4717 \\
        12 & go & 4145 \\
        13 & really & 3731 \\
        14 & also & 3333 \\
        15 & got & 3171 \\
        16 & us & 2926 \\
        17 & even & 2891 \\
        18 & order & 2820 \\
        19 & could & 2815 \\
        20 & nice & 2756 \\
    \end{tabular}
    \label{tab:most_common_tokens}
\end{table}

\begin{table}[]
    \centering
    \caption{20 most common unique stems}
    \begin{tabular}{c|c|c}
        No. & Stem & No. of Appearances  \\
        \hline
        1 & place & 9407 \\
        2 & food & 8731 \\
        3 & good & 8055 \\
        4 & time & 6588 \\
        5 & get & 6390 \\
        6 & servic & 6338 \\
        7 & great & 6302 \\
        8 & like & 6231 \\
        9 & order & 6150 \\
        10 & go & 5993 \\
        11 & one & 5278 \\
        12 & would & 5168 \\
        13 & back & 4744 \\
        14 & tri & 3978 \\
        15 & come & 3784 \\
        16 & realli & 3731 \\
        17 & also & 3333 \\
        18 & love & 3291 \\
        19 & got & 3171 \\
        20 & even & 3160\\
    \end{tabular}
    \label{tab:most_common_stems}
\end{table}
	\subsection{POS Tagging}
	\subsection{Most Frequent Adjectives for each Rating}
	\section{Development of a〈Noun - Adjective〉Pair Summarizer}
	The results are displayed in Table 5 and Table 6.
\subsection{Approach}
For the extraction of the Noun-Adjectives, we used a dependency-grammar approach. In dependency grammar, adjectives are labeled as 'amod' for sentences like "... bad service ..." while in the case of Adjectives in sentences like "... the service at ... is bad ...", the adjectives are labeled as 'acomp'. With this labels we were able to identify the adjectives. For the nouns we searched the dependency tree of the parent node of the adjective for 
the 'nsubj' relation. All of these relations were being stored in an array. After another check of the POS-Tag of the collected 'nsubj' relations, we were able to write the ADJ-NOUN Tuples to the final array. Before the adding we lemmatized the words so we can have more accurate number of the used words in the final. We decided to do so since our corpus is to small to distinct between different tenses and therefore increasing the accuracy for the summary of the used adjectives and nouns. The lemmatization was done after the Adjective and Noun were identified For the research we looked at the top twenty appearing pairs.\\
\begin{Verbatim}[fontsize=\tiny]
# the case that the adjective is a direct amod
if token.dep_ == 'amod' and token.pos_ == "ADJ":
	if token.head.pos_ == 'NOUN':

		# lemmatize the tokens
		if token.head.lemma_ == '-PRON-':
			pair = ( token.lemma_.lower(), e.text.lower() )
		else:
			pair = ( 
				token.lemma_.lower(), 
				token.head.lemma_.lower() 
				)
		pairs.append(pair)

# the case that the adjective is a adj complement
if token.dep_ == 'acomp' and token.pos_ == "ADJ":
	noun = [
		e for e in token.head.children if e.dep_ == "nsubj"
		]
	if noun:
		for e in noun:

		# skip non nouns
		if not e.pos_ == 'NOUN':
		continue

		# lemmatize the tokens
		if e.lemma_ == '-PRON-':
			pair = ( token.lemma_.lower(), e.text.lower() )
		else:
			pair = ( token.lemma_.lower(), e.lemma_.lower() )
			pairs.append(pair)
\end{Verbatim}
\subsection{Analysis}
The pairs created for Business 2 are indicating already the sector of the company. Pairs like ('clean', 'car'), ('basic', 'wash') or ('classic', 'wash') are signs, that we are dealing with the reviews of a car wash here. After reviewing the reviews, it became more obvious, that the opinions are mixed about this business. Costumers mentioned bad service, which is also indicated by the determined pairs, like ('poor', 'job'), ('terrible', 'job') or ('bad', 'service'). In contrast, there were also some good reviews about this business, which also came up in the pairs. The most common pair is, for example, ('clean', 'car'), which indicates a good service. In total the results are satisfying, since most of the pairs are providing useful information. The only exception here is ('high', 'pressure') which is too ambiguous.\\
Business 2 was also a non restaurant business. The noun-adjective pairs such as (‘front’,’desk’), (‘free’,’wifi’), (‘rental’, ‘car’), (‘complimentary’, ‘breakfast’) suggest that the reviews are about an hotel which is true. We also get a general idea from the pairs that the staff, breakfast as well as the facilities are decent which is also true from reading the reviews. Apart from some complaints here and there in the reviews, the guests generally liked the hotel and its free shuttle service. However, some pairs like (‘light’, ‘sleeper’) is useless as it talks about the guest. Also, (‘next’, ‘day’) and (‘first’, ‘day’) occur frequently which does not give us any info.\\
As for Business 3, most reviews spoke great of the dumplings as well as the food, which are the 2 most frequent noun-adjective pairs. Hence, those are accurate. As from the reviews, it is given that the restaurant has a mix of Korean as well as Northern-Chinese dishes which is also something that can be rightly perceived from the frequent noun-adjective pairs obtained, e.g, (‘korean’, ‘food’), (‘korean’, ‘dishes’), (‘chinese’, ‘cuisine’), etc. Fried pork and fried rice are some of the dishes that are sold in the restaurant that people talked of in the reviews. Most reviews talk of how bad the service is. Apparently, the management and staff is not friendly, and the food is brought in quite slowly. This is in contradiction with one of the noun-adjective pairs that we got ((‘great’), (‘service’)).  Perhaps, the noun-adjective pair could not comprehend the meaning of the sentences where service and great would appear together. These are some sentences quoted from the reviews: “If you are expecting great service, this the wrong place for you”, “but service is not so great”.  Therefore, when extracting noun-adjective pairs for these sentences, we obtain (‘great’), (‘service’), although the expression of the sentence is actually negative.\\
Business 4 represents a South Indian restaurant with a buffet set-up. Hence, the first three most frequent adj-noun pairs correctly summarizes the nature of the restaurant and the kind of food being served in the restaurant. Noun-adj pairs like (‘first’, ‘time’) do not hold much importance in the feedback of the restaurant but (‘second’, ‘time’) can refer to customers wanting to come back which speaks good of the restaurant. Besides, this is quite true because there are quite a few reviews that talks about customers wanting to give it another try. Most reviews also speak good of the staff as given by the (‘friendly’, ‘staff’) pair. Most spoken of food item is the dosa and that’s also given as one of the noun-adjective pairs. On the whole, the frequent noun-adjective pairs obtained give an accurate response of the restaurant.\\
For business 5, the determined pairs are providing useful information whereby we don't have pairs related to time. Since the second most  often pair is ('mexican', 'food') and the another pair is ('mexican', 'restaurant'), we can easy infer, that this business is a mexican restaurant. The inspection of the reviews proofed this assumption. Most of these reviews were positive towards the business. Determined paris like ('great', 'place'), ('authentic','food') or ('best','food') emphasizing this observation. Since there are many specific dishes being mentioned, the reviews are very particular. There are pairs like ('carne', 'asada'), ('toasted', 'bread') and ('iced', 'tea') which mention this particulars. The generated pairs for Business 5 are very unambiguous, such that all pairs can be used to retrieve information.
	

	
	\subsection{Conclusion}
	With this summarizer we were able to find very specific characteristics of the business. There are alot of useful adjectives to specific offers of the restaurant (e.g.: (('best', 'buffet'), 4), (('quick', 'service'), 2), (('good', 'food'), 5)). With this results we can get extract the most important pairs of the reviews. But still there are some pairs which are not useful to extract the main information of the review like time information which are not useful without their contextes (e.g. : ('single', 'time'), 2),(('first', 'night'), 3), (('first', 'day'), 2)). We assume that the reason for this is the ambigousness of 
	the POS-Tagging. Neither NLTK nor spacy where able to distinguish between the function of a determiner and an adjective. Since these time words can also be used in other contextes as adjectives, an exclusion of these words would not be reasonable. There were also some inaccurancies apperearing like  (('tomato', 'sauce'), 3) which are not usefull for the reflection.
	We can conclude, that we can extract alot of useful knowledge from the reviews using this Adjective-Noun-Summarizer. 
	
	\section{Application}
	
\end{document}
\endinput
%%
%% End of file `sample-sigchi.tex'.
 
