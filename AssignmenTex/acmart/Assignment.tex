 
%%
%% This is file `sample-sigchi.tex',
%% generated with the docstrip utility.
%%
%% The original source files were:
%%
%% samples.dtx  (with options: `sigchi')
%% 
%% IMPORTANT NOTICE:
%% 
%% For the copyright see the source file.
%% 
%% Any modified versions of this file must be renamed
%% with new filenames distinct from sample-sigchi.tex.
%% 
%% For distribution of the original source see the terms
%% for copying and modification in the file samples.dtx.
%% 
%% This generated file may be distributed as long as the
%% original source files, as listed above, are part of the
%% same distribution. (The sources need not necessarily be
%% in the same archive or directory.)
%%
%% The first command in your LaTeX source must be the \documentclass command.
\documentclass[sigchi]{acmart}

\usepackage{spverbatim}
\usepackage{graphicx}
\usepackage{longtable}
\usepackage{supertabular,booktabs}
\usepackage{spverbatim}

\usepackage{fancyvrb}
\usepackage{fvextra}

%%
%% \BibTeX command to typeset BibTeX logo in the docs
%\AtBeginDocument{%%
%	\providecommand\BibTeX{{
%			\normalfont B\kern-0.5em{\scshape i\kern-0.25em b}\kern-0.8em\TeX}}}

%% Rights management information.  This information is sent to you
%% when you complete the rights form.  These commands have SAMPLE
%% values in them; it is your responsibility as an author to replace
%% the commands and values with those provided to you when you
%% complete the rights form.

%%
%% The majority of ACM publications use numbered citations and
%% references.  The command \citestyle{authoryear} switches to the
%% "author year" style.
%%
%% If you are preparing content for an event
%% sponsored by ACM SIGGRAPH, you must use the "author year" style of
%% citations and references.
%% Uncommenting
%% the next command will enable that style.
%%\citestyle{acmauthoryear}

%%
%% end of the preamble, start of the body of the document source.
\begin{document}
	
	%%
	%% The "title" command has an optional parameter,
	%% allowing the author to define a "short title" to be used in page headers.
	\title{Natural Language Processing CZ4045}
	\subtitle{Group Report (G20C)}
	
	%%
	%% The "author" command and its associated commands are used to define
	%% the authors and their affiliations.
	%% Of note is the shared affiliation of the first two authors, and the
	%% "authornote" and "authornotemark" commands
	%% used to denote shared contribution to the research.
	\author{Philipp Koch}
	\affiliation{%
		\institution{NTU - School of Computer Science and Engineering}
		\streetaddress{21 Lien Ying Chow Drive}
		\city{637296 Singapore}
		\country{Singapore}
	}
	\email{N1903454H@e.ntu.edu.sg}
	
	\author{Gantari Evanda Raufani}
	\affiliation{%
		\institution{NTU - School of Computer Science and Engineering}
		\streetaddress{30 Nanyang Crescent}
		\city{Singapore}
		\country{Singapore}
	}
	\email{gant0010@e.ntu.edu.sg}
	
	\author{Stella Marcella}
	\affiliation{%
		\institution{NTU - School of Computer Science and Engineering}
		\streetaddress{}
		\city{Singapore}
		\country{Singapore}
	}
	\email{smarcell001@e.ntu.edu.sg}
	
	\author{Dodda Sharon Olivia}
	\affiliation{%
		\institution{NTU - School of Computer Science and Engineering}
		\streetaddress{}
		\city{Singapore}
		\country{Singapore}
	}
	\email{sharonol001@e.ntu.edu.sg}
	
	\author{Janaki H Nair}
	\affiliation{%
		\institution{NTU - School of Computer Science and Engineering}
		\streetaddress{}
		\city{Singapore}
		\country{Singapore}
	}
	\email{janakih001@e.ntu.edu.sg}
	
	%%
	%% By default, the full list of authors will be used in the page
	%% headers. Often, this list is too long, and will overlap
	%% other information printed in the page headers. This command allows
	%% the author to define a more concise list
	%% of authors' names for this purpose.
	% \renewcommand{\shortauthors}{Trovato and Tobin, et al.}
	
	%%
	%% The abstract is a short summary of the work to be presented in the
	%% article.
	\begin{abstract}
		Our task covered data processing on a dataset provided by the review platform \textit{yelp}. We had to analyze the data
		descriptively and we had to focus on the Adjectives in the reports. Therefore we had to compare different methods on how the reviews can be represented by adjectives, which also became our application model. In our application model we were able to find specific properties of the businesses reviewed in the data.
	\end{abstract}
	
	%%
	%% The code below is generated by the tool at http://dl.acm.org/ccs.cfm.
	%% Please copy and paste the code instead of the example below.
	%%

	
	%%
	%% Keywords. The author(s) should pick words that accurately describe
	%% the work being presented. Separate the keywords with commas.
	
	
	%%
	%% This command processes the author and affiliation and title
	%% information and builds the first part of the formatted document.
	\maketitle
	
	\section{Dataset Analysis}
	\subsection{Writing Style}
	For our analysis of the writing style, we picked some random sentences as a sample to compare it with articles in the well known newspaper “The Strait Times”, whereby we used the food section as a control sample. The list below shows our observations:

\begin{itemize}
    \item The last sentence does not have a final punctuation.
    \item The review lacks proper punctuation in the sentences.
    \item There is usage of words that are not in the English dictionary.
    \item There is use of slang. 
    \item The first words of the sentences in the review are not capitalized.
    \item The last sentence does not have a final punctuation.
    \item News articles are usually written in third person point of view, however the reviews from yelp are written in first person point of view.
    \item The first words of the sentences in the review are not capitalized.
    \item The last sentence does not have a final punctuation.
    \item News articles are usually written in third person point of view, however the reviews from yelp are written in first person point of view.
    \item The review contains onomatopoeia.
    \item The review uses all capitalised characters to create emphasis.
    \item News articles usually do not contain punctuation like ‘...’ that suggests pause.
    \item Words like ‘gal’ that is used to denote a girl is considered informal which is used in the review but not a news article.
    \item News articles usually use subject-verb-object construction for the sentences.
    \item The review contains many exclamation marks for one sentence although it is not necessary (known as “exclamation-point inflation” in digital communication).
    \item Some of the first words in sentences like ‘the’ are not capitalized.
    \item Personal pronouns like ‘i’ aren't capitalized.
    \item There are misspelled words, such as ‘espically’.
    \item There are missing punctuation in the reviews.
    \item The review is not relevant to the business.
    \item The review is not in full proper sentences.
    \item The review contains smileys (~2 \% of all reviews contain smileys).
\end{itemize}

	\subsection{Sentence Segmentation}
	For the sentence segmentation we used the library spacy. Each category is displayed in figure 1.
	\begin{center}
		\begin{table}[!h]
			\caption{Average length of the sentences in the different star categories}
			\begin{tabular}{ c c c c c}
				1 Star & 2 Star & 3 Star  & 4 Star & 5 Star\\
				30.5 & 25.9 & 24.0  & 25 & 24\\
			\end{tabular}
		\end{table}
	\end{center}
	
	\begin{figure}
		\caption{Histograms of the length of the sentences}
		\includegraphics[scale=0.3]{figures/1stars-Sentencelength.png}
		\includegraphics[scale=0.3]{figures/2stars-Sentencelength.png}
		\includegraphics[scale=0.3]{figures/3stars-Sentencelength.png}
		\includegraphics[scale=0.3]{figures/4stars-Sentencelength.png}
		\includegraphics[scale=0.3]{figures/5stars-Sentencelength.png}
	\end{figure}
	As can be seen, the length of sentences is on average longer with a poor rating than with a good rating. 
	
	\subsection{Tokenization and Stemming}
	For this task, we used the NLTK library to calculate the number of unique tokens that exist in a review. Then, we plotted the graphs to show the distributions of the tokens, both with and without stemming.

We can see the figure below. Figure 2 shows the number of unique tokens without stemming in every review, while Figure 3 shows the number of unique tokens after stemming is performed in every review.

\begin{figure}
    \centering
    \caption{Distributions of tokens in each review (without stemming)}
    \includegraphics[scale=0.54]{figures/token_review.png}
    \label{fig:tokenized_review}
\end{figure}

\begin{figure}
    \centering
    \caption{Distributions of tokens in each review (with stemming)}
    \includegraphics[scale=0.5]{figures/stem_review.png}
    \label{fig:stemmed_token_review}
\end{figure}

To make the comparison easier, in Figure 4, the stems are indicated with the color orange, while the tokens are indicated with the color blue.

Based on Figure 4, we can conclude that the number of unique tokens with stemming is generally lower that the number of unique tokens without performing stemming.

\begin{figure}
    \centering
    \caption{Comparison of the distributions of the tokens with and without stemming}
    \includegraphics[scale=0.5]{figures/token_stem_review.png}
    \label{fig:stem_and_token_review}
\end{figure}

\begin{table}[]
	\tiny
    \centering
    \caption{20 most common unique tokens}
    \begin{tabular}{c|c|c}
        No. & Token & No. of Appearances  \\
        \hline
        1 & food & 8580 \\
        2 & place & 8236 \\
        3 & good & 7919 \\
        4 & great & 6295 \\
        5 & service & 6027 \\
        6 & like & 5534 \\
        7 & get & 5216 \\
        8 & time & 5172 \\
        9 & would & 5168 \\
        10 & one & 5044 \\
        11 & back & 4717 \\
        12 & go & 4145 \\
        13 & really & 3731 \\
        14 & also & 3333 \\
        15 & got & 3171 \\
        16 & us & 2926 \\
        17 & even & 2891 \\
        18 & order & 2820 \\
        19 & could & 2815 \\
        20 & nice & 2756 \\
    \end{tabular}
    \label{tab:most_common_tokens}
\end{table}

\begin{table}[]
    \centering
    \caption{20 most common unique stems}
    \begin{tabular}{c|c|c}
        No. & Stem & No. of Appearances  \\
        \hline
        1 & place & 9407 \\
        2 & food & 8731 \\
        3 & good & 8055 \\
        4 & time & 6588 \\
        5 & get & 6390 \\
        6 & servic & 6338 \\
        7 & great & 6302 \\
        8 & like & 6231 \\
        9 & order & 6150 \\
        10 & go & 5993 \\
        11 & one & 5278 \\
        12 & would & 5168 \\
        13 & back & 4744 \\
        14 & tri & 3978 \\
        15 & come & 3784 \\
        16 & realli & 3731 \\
        17 & also & 3333 \\
        18 & love & 3291 \\
        19 & got & 3171 \\
        20 & even & 3160\\
    \end{tabular}
    \label{tab:most_common_stems}
\end{table}
	\subsection{POS Tagging}
	For the POS tagging, we used the word\_tokenize() function from the nltk library. The table below shows the results of POS tagging 5 random reviews.

\setlength{\tabcolsep}{18pt}
\renewcommand{\arraystretch}{1.75}
    \begin{longtable}{|| p{3cm} | p{3cm} ||} 
        \hline 
         \textbf{Review/Text} & \textbf{Parts of Speech} \\ [0.5ex] 
         \hline\hline
         The bubble tea was excellent! The service however, was not. I found it pushy, rushed, abrasive, and incredibly rude. Their way of dealing with English speaking customers is to bark at them until they leave with their order. You certainly won't be receiving any of my money ever again. & [('The', 'DT'), ('bubble', 'JJ'), ('tea', 'NN'), ('was', 'VBD'), ('excellent', 'JJ'), ('!', '.'), ('The', 'DT'), ('service', 'NN'), ('however', 'RB'), (',', ','), ('was', 'VBD'), ('not', 'RB'), ('.', '.'), ('I', 'PRP'), ('found', 'VBD'), ('it', 'PRP'), ('pushy', 'VBZ'), (',', ','), ('rushed', 'VBD'), (',', ','), ('abrasive', 'JJ'), (',', ','), ('and', 'CC'), ('incredibly', 'RB'), ('rude', 'VB'), ('.', '.'), ('Their', 'PRP\$'), ('way', 'NN'), ('of', 'IN'), ('dealing', 'VBG'), ('with', 'IN'), ('English', 'NNP'), ('speaking', 'VBG'), ('customers', 'NNS'), ('is', 'VBZ'), ('to', 'TO'), ('bark', 'VB'), ('at', 'IN'), ('them', 'PRP'), ('until', 'IN'), ('they', 'PRP'), ('leave', 'VBP'), ('with', 'IN'), ('their', 'PRP\$'), ('order', 'NN'), ('.', '.'), ('You', 'PRP'), ('certainly', 'RB'), ('wo', 'MD'), ("n't", 'RB'), ('be', 'VB'), ('receiving', 'VBG'), ('any', 'DT'), ('of', 'IN'), ('my', 'PRP\$'), ('money', 'NN'), ('ever', 'RB'), ('again', 'RB'), ('.', '.')] \\ 
         \hline
         I love the sketch comedy nights! Food is not so great. Drinks are average prices. Good venue. & [('I', 'PRP'), ('love', 'VBP'), ('the', 'DT'), ('sketch', 'NN'), ('comedy', 'NN'), ('nights', 'NNS'), ('!', '.'), ('Food', 'NNP'), ('is', 'VBZ'), ('not', 'RB'), ('so', 'RB'), ('great', 'JJ'), ('.', '.'), ('Drinks', 'NNS'), ('are', 'VBP'), ('average', 'JJ'), ('prices', 'NNS'), ('.', '.'), ('Good', 'JJ'), ('venue', 'NN'), ('.', '.')] \\
         \hline
         This place is inside Baiz Supermarket. If you are in mood of middle eastern food, this is the place. They have the most popular dishes. I highly recommend this place! & [('This', 'DT'), ('place', 'NN'), ('is', 'VBZ'), ('inside', 'JJ'), ('Baiz', 'NNP'), ('Supermarket', 'NNP'), ('.', '.'), ('If', 'IN'), ('you', 'PRP'), ('are', 'VBP'), ('in', 'IN'), ('mood', 'NN'), ('of', 'IN'), ('middle', 'JJ'), ('eastern', 'JJ'), ('food', 'NN'), (',', ','), ('this', 'DT'), ('is', 'VBZ'), ('the', 'DT'), ('place', 'NN'), ('.', '.'), ('They', 'PRP'), ('have', 'VBP'), ('the', 'DT'), ('most', 'RBS'), ('popular', 'JJ'), ('dishes', 'NNS'), ('.', '.'), ('I', 'PRP'), ('highly', 'RB'), ('recommend', 'VBP'), ('this', 'DT'), ('place', 'NN'), ('!', '.')] \\
         \hline
         I used to love Jukebox but after today's meal, all I can say is "it was nice while it lasted". That was the dryest, most flavorless burger I've ever had. The cook didn't even bother getting the patty all on the bun, but just slapped everything together in a sloppy, careless mess. The fries were way overcooked (it used to come with curly but those are now an extra \$\$). The bun was plain, untoasted and there was no tomato (there was supposed to be). \$18 for overpriced cardboard. & [('I', 'PRP'), ('used', 'VBD'), ('to', 'TO'), ('love', 'VB'), ('Jukebox', 'NNP'), ('but', 'CC'), ('after', 'IN'), ('today', 'NN'), ("'s", 'POS'), ('meal', 'NN'), (',', ','), ('all', 'DT'), ('I', 'PRP'), ('can', 'MD'), ('say', 'VB'), ('is', 'VBZ'), ('``', '``'), ('it', 'PRP'), ('was', 'VBD'), ('nice', 'RB'), ('while', 'IN'), ('it', 'PRP'), ('lasted', 'VBD'), ("''", "''"), ('.', '.'), ('That', 'DT'), ('was', 'VBD'), ('the', 'DT'), ('dryest', 'JJS'), (',', ','), ('most', 'JJS'), ('flavorless', 'JJ'), ('burger', 'NN'), ('I', 'PRP'), ("'ve", 'VBP'), ('ever', 'RB'), ('had', 'VBN'), ('.', '.'), ('The', 'DT'), ('cook', 'NN'), ('did', 'VBD'), ("n't", 'RB'), ('even', 'RB'), ('bother', 'VB'), ('getting', 'VBG'), ('the', 'DT'), ('patty', 'NN'), ('all', 'DT'), ('on', 'IN'), ('the', 'DT'), ('bun', 'NN'), (',', ','), ('but', 'CC'), ('just', 'RB'), ('slapped', 'VBD'), ('everything', 'NN'), ('together', 'RB'), ('in', 'IN'), ('a', 'DT'), ('sloppy', 'JJ'), (',', ','), ('careless', 'JJ'), ('mess', 'NN'), ('.', '.'), ('The', 'DT'), ('fries', 'NNS'), ('were', 'VBD'), ('way', 'NN'), ('overcooked', 'VBN'), ('(', '('), ('it', 'PRP'), ('used', 'VBD'), ('to', 'TO'), ('come', 'VB'), ('with', 'IN'), ('curly', 'JJ'), ('but', 'CC'), ('those', 'DT'), ('are', 'VBP'), ('now', 'RB'), ('an', 'DT'), ('extra', 'JJ'), ('\$', '\$'), ('\$', '\$'), (')', ')'), ('.', '.'), ('The', 'DT'), ('bun', 'NN'), ('was', 'VBD'), ('plain', 'RB'), (',', ','), ('untoasted', 'JJ'), ('and', 'CC'), ('there', 'EX'), ('was', 'VBD'), ('no', 'DT'), ('tomato', 'NN'), ('(', '('), ('there', 'EX'), ('was', 'VBD'), ('supposed', 'VBN'), ('to', 'TO'), ('be', 'VB'), (')', ')'), ('.', '.'), ('\$', '\$'), ('18', 'CD'), ('for', 'IN'), ('overpriced', 'VBN'), ('cardboard', 'NN'), ('.', '.')] \\
         \hline
         Scheduled manicure and pedicure one day. Came in, was seated and was told to wait. Waited for over 30 minutes, went back to the front desk, they said they forgot about me. I was nice about it, I guess that happenes (not at the good places, but still happens). At the end wasn't even offered a discount... pretty shitty move. & [('Scheduled', 'VBN'), ('manicure', 'NN'), ('and', 'CC'), ('pedicure', 'NN'), ('one', 'CD'), ('day', 'NN'), ('.', '.'), ('Came', 'NN'), ('in', 'IN'), (',', ','), ('was', 'VBD'), ('seated', 'VBN'), ('and', 'CC'), ('was', 'VBD'), ('told', 'VBN'), ('to', 'TO'), ('wait', 'VB'), ('.', '.'), ('Waited', 'VBN'), ('for', 'IN'), ('over', 'IN'), ('30', 'CD'), ('minutes', 'NNS'), (',', ','), ('went', 'VBD'), ('back', 'RB'), ('to', 'TO'), ('the', 'DT'), ('front', 'NN'), ('desk', 'NN'), (',', ','), ('they', 'PRP'), ('said', 'VBD'), ('they', 'PRP'), ('forgot', 'VBD'), ('about', 'IN'), ('me', 'PRP'), ('.', '.'), ('I', 'PRP'), ('was', 'VBD'), ('nice', 'JJ'), ('about', 'IN'), ('it', 'PRP'), (',', ','), ('I', 'PRP'), ('guess', 'VBP'), ('that', 'IN'), ('happenes', 'NNS'), ('(', '('), ('not', 'RB'), ('at', 'IN'), ('the', 'DT'), ('good', 'JJ'), ('places', 'NNS'), (',', ','), ('but', 'CC'), ('still', 'RB'), ('happens', 'VBZ'), (')', ')'), ('.', '.'), ('At', 'IN'), ('the', 'DT'), ('end', 'NN'), ('was', 'VBD'), ("n't", 'RB'), ('even', 'RB'), ('offered', 'VBN'), ('a', 'DT'), ('discount', 'NN'), ('...', ':'), ('pretty', 'RB'), ('shitty', 'JJ'), ('move', 'NN'), ('.', '.')] \\  
         \hline
    \end{longtable}

    


	\subsection{Most Frequent Adjectives for each Rating}
	There are two tasks in this section: (1) To list the top-10 most frequently used adjectives for each rating star (i.e. 1 to 5), and (2) To list the top-10 most indicative adjectives for each rating star. First, we tokenize all the words (including punctuations) in all the reviews and count the frequency of each word in all the reviews, storing both in separate arrays. Then, we separate the reviews based on the rating star, tokenize the reviews and do POS tagging on the tokens to identify the adjectives. Using the nltk library, the POS tagging for adjectives is 'JJ'. For each adjective identified in a rating star, the adjective and its frequency will be stored in two arrays. After going through all the reviews with the particular rating star, we calculate the "indicativeness" of each adjectives using the formula:
    \begin{equation}
        Indicativeness = P(w|Ri)×log(P(w|Ri)/P(w))
    \end{equation}
    
    where P(w|Ri) is the probability of observing the adjective w in the reviews with rating star i, and P(w) is the probability of observing the adjective w in all the reviews.
    
The adjectives array is then sorted twice, based on the frequency and based on the indicativeness. From the sorted arrays, we return the top-10 most frequently used adjectives for the rating star and the top-10 most indicative adjectives for the rating star, and print them. The result is as follow:

    \begin{center}
        \tiny
        \begin{table}[!h]
        \caption{Top 10 Most Frequently Used Adjectives}
            \begin{tabular}{c c c c c}
                1 Star & 2 Stars & 3 Stars & 4 Stars & 5 Stars\\
                'good' '661' & 'good' '789' & 'good' '1552' & 'good' '2711' & 'great' '2761'\\
                'other' '543' & 'other' '396' & 'other' '570' & 'great' '1572' & 'good' '2099'\\
                'bad' '433' & 'great' '289' & 'great' '493' & 'nice' '793' & 'friendly' '974'\\
                'first' '337' & 'bad' '236' & 'nice' '430' & 'other' '782' & 'delicious' '913'\\
                'new' '319' & 'nice' '215' & 'little' '384' & 'little' '693' & 'nice' '849'\\
                'last' '310' & 'first' '214' & 'small' '285' & 'delicious''616' & 'other' '800'\\
                'horrible' '257' & 'much' '202' & 'bad' '267' & 'fresh' '465' & 'amazing' '645'\\
                'few' '249' & 'little' '193' & 'few' '250' & 'friendly' '462' & 'fresh' '643'\\
                'same' '246' & 'few' '181' & 'decent' '237' & 'small' '427' & 'first' '622'\\
                'next' '244' & 'small' '175' & 'hot' '200' & 'first' '392' & 'new' '607'\\
                 
            \end{tabular}
        \end{table}
    \end{center}
    
    \begin{center}
        \tiny
        \begin{table}[!h]
        \caption{Top 10 Most Indicative Adjectives}
            \begin{tabular}{c c c c c}
                1 Star & 2 Stars\\
                'horrible' '0.000726' & 'bad' '0.000337'\\
                'terrible' '0.000669' & 'same' '0.000297'\\
                'bad' '0.000458' & 'dry' '0.000237'\\
                'rude' '0.000381' & 'other' '0.000165'\\
                'poor' '0.000289' & 'second' '0.000161'\\
                'awful' '0.000262' & 'decent' '0.000160'\\
                'last' '0.000230' & 'chinese' '0.000157'\\
                'unprofessional' '0.000194' & 'slow' '0.000153'\\
                'same' '0.000149' & 'empty' '0.000149'\\
                'wrong' '0.000127' & 'ok' '0.000143'\\
                \\
            \end{tabular}
            
            \begin{tabular}{c c c}
                3 Stars & 4 Stars & 5 Stars\\
                'good' '0.001890' & 'good' '0.002338' & 'great' '0.001498'\\
                'decent' '0.000767' & 'little' '0.000473' & 'delicious' '0.000638'\\
                'other' '0.000563' & 'delicious' '0.000463' & 'happy' '0.000513'\\
                'small' '0.000448' & 'nice' '0.000379' & 'friendly' '0.000470'\\
                'ok' '0.000350' & 'tasty' '0.000365' & 'wonderful' '0.000396'\\
                'little' '0.000308' & 'small' '0.000315' & 'excellent' '0.000382'\\
                'average' '0.000287' & 'fresh' '0.000264' & 'professional' '0.000370'\\
                'bad' '0.000252' & 'great' '0.000258' & 'helpful' '0.000358'\\
                'large' '0.000244' & 'busy' '0.000190' & 'fantastic' '0.000353'\\
                'fine' '0.000190' & 'large' '0.000188' & 'awesome' '0.000338'\\
            \end{tabular}
        \end{table}
    \end{center}

The results for the most indicative adjectives are more descriptive for a rating system where 1 stars are the worst rating. It gives more weight to specific words (terrible, horrible, unprofessional) which appear more frequently in the specific rating. Meanwhile, the list of most frequently used adjectives is less descriptive since it does not take into account for negation. As an example, the following table shows the frequency of the adjective 'good' in each rating star.

    \begin{center}
        \begin{table}[!h]
        \caption{Frequency of the word 'good' in each rating star}
            \begin{tabular}{c c c c c}
                1 Star & 2 Stars & 3 Stars & 4 Stars & 5 Stars\\
                661 & 789 & 1552 & 2711 & 2099\\
            \end{tabular}
        \end{table}
    \end{center}
    
Although the word 'good' is in the top 10 list of most frequently used adjectives for all the rating star, the word 'good' in reviews with lower number of stars is more likely to refer to 'not good' than in reviews with higher number of stars. The result may be less accurate because in the tokenization, the punctuations are also counted and are included in the word count. We assume that this will not affect the result to a large extent because punctuations make up a very small proportion of all the words in all the reviews. We conclude that the most indicative adjectives are more useful to identify the different rating stars.
	\section{Development of a <Noun - Adjective> Pair Summarizer}
	The results are displayed in Table 5 and Table 6.
\subsection{Approach}
For the extraction of the Noun-Adjectives, we used a dependency-grammar approach. In dependency grammar, adjectives are labeled as 'amod' for sentences like "... bad service ..." while in the case of Adjectives in sentences like "... the service at ... is bad ...", the adjectives are labeled as 'acomp'. With this labels we were able to identify the adjectives. For the nouns we searched the dependency tree of the parent node of the adjective for 
the 'nsubj' relation. All of these relations were being stored in an array. After another check of the POS-Tag of the collected 'nsubj' relations, we were able to write the ADJ-NOUN Tuples to the final array. Before the adding we lemmatized the words so we can have more accurate number of the used words in the final. We decided to do so since our corpus is to small to distinct between different tenses and therefore increasing the accuracy for the summary of the used adjectives and nouns. The lemmatization was done after the Adjective and Noun were identified For the research we looked at the top twenty appearing pairs.\\
\begin{Verbatim}[fontsize=\tiny]
# the case that the adjective is a direct amod
if token.dep_ == 'amod' and token.pos_ == "ADJ":
	if token.head.pos_ == 'NOUN':

		# lemmatize the tokens
		if token.head.lemma_ == '-PRON-':
			pair = ( token.lemma_.lower(), e.text.lower() )
		else:
			pair = ( 
				token.lemma_.lower(), 
				token.head.lemma_.lower() 
				)
		pairs.append(pair)

# the case that the adjective is a adj complement
if token.dep_ == 'acomp' and token.pos_ == "ADJ":
	noun = [
		e for e in token.head.children if e.dep_ == "nsubj"
		]
	if noun:
		for e in noun:

		# skip non nouns
		if not e.pos_ == 'NOUN':
		continue

		# lemmatize the tokens
		if e.lemma_ == '-PRON-':
			pair = ( token.lemma_.lower(), e.text.lower() )
		else:
			pair = ( token.lemma_.lower(), e.lemma_.lower() )
			pairs.append(pair)
\end{Verbatim}
\subsection{Analysis}
The pairs created for Business 2 are indicating already the sector of the company. Pairs like ('clean', 'car'), ('basic', 'wash') or ('classic', 'wash') are signs, that we are dealing with the reviews of a car wash here. After reviewing the reviews, it became more obvious, that the opinions are mixed about this business. Costumers mentioned bad service, which is also indicated by the determined pairs, like ('poor', 'job'), ('terrible', 'job') or ('bad', 'service'). In contrast, there were also some good reviews about this business, which also came up in the pairs. The most common pair is, for example, ('clean', 'car'), which indicates a good service. In total the results are satisfying, since most of the pairs are providing useful information. The only exception here is ('high', 'pressure') which is too ambiguous.\\
Business 2 was also a non restaurant business. The noun-adjective pairs such as (‘front’,’desk’), (‘free’,’wifi’), (‘rental’, ‘car’), (‘complimentary’, ‘breakfast’) suggest that the reviews are about an hotel which is true. We also get a general idea from the pairs that the staff, breakfast as well as the facilities are decent which is also true from reading the reviews. Apart from some complaints here and there in the reviews, the guests generally liked the hotel and its free shuttle service. However, some pairs like (‘light’, ‘sleeper’) is useless as it talks about the guest. Also, (‘next’, ‘day’) and (‘first’, ‘day’) occur frequently which does not give us any info.\\
As for Business 3, most reviews spoke great of the dumplings as well as the food, which are the 2 most frequent noun-adjective pairs. Hence, those are accurate. As from the reviews, it is given that the restaurant has a mix of Korean as well as Northern-Chinese dishes which is also something that can be rightly perceived from the frequent noun-adjective pairs obtained, e.g, (‘korean’, ‘food’), (‘korean’, ‘dishes’), (‘chinese’, ‘cuisine’), etc. Fried pork and fried rice are some of the dishes that are sold in the restaurant that people talked of in the reviews. Most reviews talk of how bad the service is. Apparently, the management and staff is not friendly, and the food is brought in quite slowly. This is in contradiction with one of the noun-adjective pairs that we got ((‘great’), (‘service’)).  Perhaps, the noun-adjective pair could not comprehend the meaning of the sentences where service and great would appear together. These are some sentences quoted from the reviews: “If you are expecting great service, this the wrong place for you”, “but service is not so great”.  Therefore, when extracting noun-adjective pairs for these sentences, we obtain (‘great’), (‘service’), although the expression of the sentence is actually negative.\\
Business 4 represents a South Indian restaurant with a buffet set-up. Hence, the first three most frequent adj-noun pairs correctly summarizes the nature of the restaurant and the kind of food being served in the restaurant. Noun-adj pairs like (‘first’, ‘time’) do not hold much importance in the feedback of the restaurant but (‘second’, ‘time’) can refer to customers wanting to come back which speaks good of the restaurant. Besides, this is quite true because there are quite a few reviews that talks about customers wanting to give it another try. Most reviews also speak good of the staff as given by the (‘friendly’, ‘staff’) pair. Most spoken of food item is the dosa and that’s also given as one of the noun-adjective pairs. On the whole, the frequent noun-adjective pairs obtained give an accurate response of the restaurant.\\
For business 5, the determined pairs are providing useful information whereby we don't have pairs related to time. Since the second most  often pair is ('mexican', 'food') and the another pair is ('mexican', 'restaurant'), we can easy infer, that this business is a mexican restaurant. The inspection of the reviews proofed this assumption. Most of these reviews were positive towards the business. Determined paris like ('great', 'place'), ('authentic','food') or ('best','food') emphasizing this observation. Since there are many specific dishes being mentioned, the reviews are very particular. There are pairs like ('carne', 'asada'), ('toasted', 'bread') and ('iced', 'tea') which mention this particulars. The generated pairs for Business 5 are very unambiguous, such that all pairs can be used to retrieve information.
	

	
	\subsection{Conclusion}
	With this summarizer we were able to find very specific characteristics of the business. There are alot of useful adjectives to specific offers of the restaurant (e.g.: (('best', 'buffet'), 4), (('quick', 'service'), 2), (('good', 'food'), 5)). With this results we can get extract the most important pairs of the reviews. But still there are some pairs which are not useful to extract the main information of the review like time information which are not useful without their contextes (e.g. : ('single', 'time'), 2),(('first', 'night'), 3), (('first', 'day'), 2)). We assume that the reason for this is the ambigousness of 
	the POS-Tagging. Neither NLTK nor spacy where able to distinguish between the function of a determiner and an adjective. Since these time words can also be used in other contextes as adjectives, an exclusion of these words would not be reasonable. There were also some inaccurancies apperearing like  (('tomato', 'sauce'), 3) which are not usefull for the reflection.
	We can conclude, that we can extract alot of useful knowledge from the reviews using this Adjective-Noun-Summarizer. 
	
	\section{Application}
	Based on the most indicatives adjectives for each rating star found in the previous part. We develop an application to identify the good and bad features of a business. Taking in a user input for the business id, the application will print out the list of good features and bad features of the particular business. This is done by first, making lists of adjectives with positive connotation and negative connotation. We also make another list containing negative words that negates the meaning of an adjective, such as no, not, never, none. 

For each review of a particular business, which contains any of the positive or negative adjectives in the list, we check if the next word is a noun. If the next word is a noun, we append the adjective-noun to either list of good features or list of bad features, based on the nature of the adjective identified. On the other hand, if the next word is not a noun, we check if the previous word is a negation, if it is, we will append the previous three words to the adjectives to either list of good features or list of bad features. For example, if a review contains the phrase: 'the service was not great at all.', and 'great' is in our list of good adjective, then we will append 'service was not great' to the list of bad features because it the word before 'great' is 'not', a negation word. The results can be found in the appendix, Table 7.

From the result, we can deduce that the business is a restaurant with a lot of positive reviews stating that it serves 'delicious food' and 'fresh fruit', although there are also some negative reviews stating that the 'service was not great' and 'poor job'. The application helps business owner and customers alike to identify more specific features available at the business. In this case, the restaurant is known for serving 'fresh fruit' and has various delicious menu, such as pork, sandwich, and chicken.

	\newpage
	\section{Appendix}
	\begin{figure}[!h]
	\caption{Histograms of the length of the sentences}
	\includegraphics[scale=0.3]{figures/1stars-Sentencelength.png}
	\includegraphics[scale=0.3]{figures/2stars-Sentencelength.png}
	\includegraphics[scale=0.3]{figures/3stars-Sentencelength.png}
	\includegraphics[scale=0.3]{figures/4stars-Sentencelength.png}
	\includegraphics[scale=0.3]{figures/5stars-Sentencelength.png}
\end{figure}
\newpage
\begin{figure}[!h]
	\centering
	\caption{Distributions of tokens in each review (without stemming)}
	\includegraphics[scale=0.54]{figures/token_review.png}
	%\label{fig:tokenized_review}
\end{figure}

\begin{figure}[!h]
	\centering
	\caption{Distributions of tokens in each review (with stemming)}
	\includegraphics[scale=0.5]{figures/stem_review.png}
	%\label{fig:stemmed_token_review}
\end{figure}
\newpage
\begin{figure}[!h]
	\centering
	\caption{Comparison of the distributions of the tokens with and without stemming}
	\includegraphics[scale=0.5]{figures/token_stem_review.png}
	%s\label{fig:stem_and_token_review}
\end{figure}

\begin{table}[!h]
	\tiny
	\centering
	\caption{20 most common unique tokens}
	\begin{tabular}{c|c|c}
		No. & Token & No. of Appearances  \\
		\hline
		1 & food & 8580 \\
		2 & place & 8236 \\
		3 & good & 7919 \\
		4 & great & 6295 \\
		5 & service & 6027 \\
		6 & like & 5534 \\
		7 & get & 5216 \\
		8 & time & 5172 \\
		9 & would & 5168 \\
		10 & one & 5044 \\
		11 & back & 4717 \\
		12 & go & 4145 \\
		13 & really & 3731 \\
		14 & also & 3333 \\
		15 & got & 3171 \\
		16 & us & 2926 \\
		17 & even & 2891 \\
		18 & order & 2820 \\
		19 & could & 2815 \\
		20 & nice & 2756 \\
	\end{tabular}
	%\label{tab:most_common_tokens}
\end{table}

\begin{table}[!h]
	\tiny
	\centering
	\caption{20 most common unique stems}
	\begin{tabular}{c|c|c}
		No. & Stem & No. of Appearances  \\
		\hline
		1 & place & 9407 \\
		2 & food & 8731 \\
		3 & good & 8055 \\
		4 & time & 6588 \\
		5 & get & 6390 \\
		6 & servic & 6338 \\
		7 & great & 6302 \\
		8 & like & 6231 \\
		9 & order & 6150 \\
		10 & go & 5993 \\
		11 & one & 5278 \\
		12 & would & 5168 \\
		13 & back & 4744 \\
		14 & tri & 3978 \\
		15 & come & 3784 \\
		16 & realli & 3731 \\
		17 & also & 3333 \\
		18 & love & 3291 \\
		19 & got & 3171 \\
		20 & even & 3160\\
	\end{tabular}
	%\label{tab:most_common_stems}
\end{table}
\newpage
\captionof{table}{POS-Tags}
\begin{Verbatim}[breaklines=true, breakanywhere=true]
[('The', 'DT'), ('bubble', 'JJ'), ('tea', 'NN'), ('was', 'VBD'), ('excellent', 'JJ'), ('!', '.'), ('The', 'DT'), ('service', 'NN'), ('however', 'RB'), (',', ','), ('was', 'VBD'), ('not', 'RB'), ('.', '.'), ('I', 'PRP'), ('found', 'VBD'), ('it', 'PRP'), ('pushy', 'VBZ'), (',', ','), ('rushed', 'VBD'), (',', ','), ('abrasive', 'JJ'), (',', ','), ('and', 'CC'), ('incredibly', 'RB'), ('rude', 'VB'), ('.', '.'), ('Their', 'PRP\$'), ('way', 'NN'), ('of', 'IN'), ('dealing', 'VBG'), ('with', 'IN'), ('English', 'NNP'), ('speaking', 'VBG'), ('customers', 'NNS'), ('is', 'VBZ'), ('to', 'TO'), ('bark', 'VB'), ('at', 'IN'), ('them', 'PRP'), ('until', 'IN'), ('they', 'PRP'), ('leave', 'VBP'), ('with', 'IN'), ('their', 'PRP\$'), ('order', 'NN'), ('.', '.'), ('You', 'PRP'), ('certainly', 'RB'), ('wo', 'MD'), ("n't", 'RB'), ('be', 'VB'), ('receiving', 'VBG'), ('any', 'DT'), ('of', 'IN'), ('my', 'PRP\$'), ('money', 'NN'), ('ever', 'RB'), ('again', 'RB'), ('.', '.')]
\end{Verbatim}

\textit{I love the sketch comedy nights! Food is not so great. Drinks are average prices. Good venue.}\\
\begin{Verbatim}[breaklines=true, breakanywhere=true] 
[('I', 'PRP'), ('love', 'VBP'), ('the', 'DT'), ('sketch', 'NN'), ('comedy', 'NN'), ('nights', 'NNS'), ('!', '.'), ('Food', 'NNP'), ('is', 'VBZ'), ('not', 'RB'), ('so', 'RB'), ('great', 'JJ'), ('.', '.'), ('Drinks', 'NNS'), ('are', 'VBP'), ('average', 'JJ'), ('prices', 'NNS'), ('.', '.'), ('Good', 'JJ'), ('venue', 'NN'), ('.', '.')]
\end{Verbatim}

\textit{This place is inside Baiz Supermarket. If you are in mood of middle eastern food, this is the place. They have the most popular dishes. I highly recommend this place!}\\
\begin{Verbatim}[breaklines=true, breakanywhere=true] 
[('This', 'DT'), ('place', 'NN'), ('is', 'VBZ'), ('inside', 'JJ'), ('Baiz', 'NNP'), ('Supermarket', 'NNP'), ('.', '.'), ('If', 'IN'), ('you', 'PRP'), ('are', 'VBP'), ('in', 'IN'), ('mood', 'NN'), ('of', 'IN'), ('middle', 'JJ'), ('eastern', 'JJ'), ('food', 'NN'), (',', ','), ('this', 'DT'), ('is', 'VBZ'), ('the', 'DT'), ('place', 'NN'), ('.', '.'), ('They', 'PRP'), ('have', 'VBP'), ('the', 'DT'), ('most', 'RBS'), ('popular', 'JJ'), ('dishes', 'NNS'), ('.', '.'), ('I', 'PRP'), ('highly', 'RB'), ('recommend', 'VBP'), ('this', 'DT'), ('place', 'NN'), ('!', '.')]
\end{Verbatim}

\textit{Scheduled manicure and pedicure one day. Came in, was seated and was told to wait. Waited for over 30 minutes, went back to the front desk, they said they forgot about me. I was nice about it, I guess that happenes (not at the good places, but still happens). At the end wasn't even offered a discount... pretty shitty move.}\\

\begin{Verbatim}[breaklines=true, breakanywhere=true] 
[('Scheduled', 'VBN'), ('manicure', 'NN'), ('and', 'CC'), ('pedicure', 'NN'), ('one', 'CD'), ('day', 'NN'), ('.', '.'), ('Came', 'NN'), ('in', 'IN'), (',', ','), ('was', 'VBD'), ('seated', 'VBN'), ('and', 'CC'), ('was', 'VBD'), ('told', 'VBN'), ('to', 'TO'), ('wait', 'VB'), ('.', '.'), ('Waited', 'VBN'), ('for', 'IN'), ('over', 'IN'), ('30', 'CD'), ('minutes', 'NNS'), (',', ','), ('went', 'VBD'), ('back', 'RB'), ('to', 'TO'), ('the', 'DT'), ('front', 'NN'), ('desk', 'NN'), (',', ','), ('they', 'PRP'), ('said', 'VBD'), ('they', 'PRP'), ('forgot', 'VBD'), ('about', 'IN'), ('me', 'PRP'), ('.', '.'), ('I', 'PRP'), ('was', 'VBD'), ('nice', 'JJ'), ('about', 'IN'), ('it', 'PRP'), (',', ','), ('I', 'PRP'), ('guess', 'VBP'), ('that', 'IN'), ('happenes', 'NNS'), ('(', '('), ('not', 'RB'), ('at', 'IN'), ('the', 'DT'), ('good', 'JJ'), ('places', 'NNS'), (',', ','), ('but', 'CC'), ('still', 'RB'), ('happens', 'VBZ'), (')', ')'), ('.', '.'), ('At', 'IN'), ('the', 'DT'), ('end', 'NN'), ('was', 'VBD'), ("n't", 'RB'), ('even', 'RB'), ('offered', 'VBN'), ('a', 'DT'), ('discount', 'NN'), ('...', ':'), ('pretty', 'RB'), ('shitty', 'JJ'), ('move', 'NN'), ('.', '.')]
\end{Verbatim}

\textit{I used to love Jukebox but after today's meal, all I can say is "it was nice while it lasted". That was the dryest, most flavorless burger I've ever had. The cook didn't even bother getting the patty all on the bun, but just slapped everything together in a sloppy, careless mess. The fries were way overcooked (it used to come with curly but those are now an extra \$\$). The bun was plain, untoasted and there was no tomato (there was supposed to be). \$18 for overpriced cardboard.}\\
\begin{Verbatim}[breaklines=true, breakanywhere=true] 
[('I', 'PRP'), ('used', 'VBD'), ('to', 'TO'), ('love', 'VB'), ('Jukebox', 'NNP'), ('but', 'CC'), ('after', 'IN'), ('today', 'NN'), ("'s", 'POS'), ('meal', 'NN'), (',', ','), ('all', 'DT'), ('I', 'PRP'), ('can', 'MD'), ('say', 'VB'), ('is', 'VBZ'), ('``', '``'), ('it', 'PRP'), ('was', 'VBD'), ('nice', 'RB'), ('while', 'IN'), ('it', 'PRP'), ('lasted', 'VBD'), ("''", "''"), ('.', '.'), ('That', 'DT'), ('was', 'VBD'), ('the', 'DT'), ('dryest', 'JJS'), (',', ','), ('most', 'JJS'), ('flavorless', 'JJ'), ('burger', 'NN'), ('I', 'PRP'), ("'ve", 'VBP'), ('ever', 'RB'), ('had', 'VBN'), ('.', '.'), ('The', 'DT'), ('cook', 'NN'), ('did', 'VBD'), ("n't", 'RB'), ('even', 'RB'), ('bother', 'VB'), ('getting', 'VBG'), ('the', 'DT'), ('patty', 'NN'), ('all', 'DT'), ('on', 'IN'), ('the', 'DT'), ('bun', 'NN'), (',', ','), ('but', 'CC'), ('just', 'RB'), ('slapped', 'VBD'), ('everything', 'NN'), ('together', 'RB'), ('in', 'IN'), ('a', 'DT'), ('sloppy', 'JJ'), (',', ','), ('careless', 'JJ'), ('mess', 'NN'), ('.', '.'), ('The', 'DT'), ('fries', 'NNS'), ('were', 'VBD'), ('way', 'NN'), ('overcooked', 'VBN'), ('(', '('), ('it', 'PRP'), ('used', 'VBD'), ('to', 'TO'), ('come', 'VB'), ('with', 'IN'), ('curly', 'JJ'), ('but', 'CC'), ('those', 'DT'), ('are', 'VBP'), ('now', 'RB'), ('an', 'DT'), ('extra', 'JJ'), ('\$', '\$'), ('\$', '\$'), (')', ')'), ('.', '.'), ('The', 'DT'), ('bun', 'NN'), ('was', 'VBD'), ('plain', 'RB'), (',', ','), ('untoasted', 'JJ'), ('and', 'CC'), ('there', 'EX'), ('was', 'VBD'), ('no', 'DT'), ('tomato', 'NN'), ('(', '('), ('there', 'EX'), ('was', 'VBD'), ('supposed', 'VBN'), ('to', 'TO'), ('be', 'VB'), (')', ')'), ('.', '.'), ('\$', '\$'), ('18', 'CD'), ('for', 'IN'), ('overpriced', 'VBN'), ('cardboard', 'NN'), ('.', '.')]
\end{Verbatim}
\newpage
	\begin{center}
	\tiny
	\begin{table}[!h]
		\caption{Business reviews}
		\begin{tabular}{c | c}
			\textbf{Business 1}			&\textbf{Business 2}\\ \hline
			2xrpo-LXV9uGIwpvy0dwUw		&oLb3-eXUFtCFJl2DuBhcvA\\\hline
			(('clean', 'car'), 4)		&(('front', 'desk'), 20)\\
			(('other', 'locations'), 3)		&(('free', 'wifi'), 5)\\
			(('great', 'job'), 3)		&(('clean', 'room'), 5)\\
			(('basic', 'wash'), 2)		&(('next', 'day'), 5)\\
			(('terrible', 'wash'), 2)		&(('next', 'door'), 4)\\
			(('poor', 'job'), 2)		&(('light', 'sleeper'), 4)\\
			(('high', 'pressure'), 2)		&(('rental', 'car'), 3)\\
			(('terrible', 'job'), 2)		&(('friendly', 'staff'), 3)\\
			(('helpful', 'guy'), 2)		&(('comfortable', 'bed'), 3)\\
			(('horrible', 'service'), 2)		&(('great', 'breakfast'), 3)\\
			(('bad', 'service'), 2)		&(('hot', 'food'), 3)\\
			(('worth', 'place'), 2)		&(('free', 'breakfast'), 3)\\
			(('different', 'options'), 2)		&(('continental', 'breakfast'), 3)\\
			(('only', 'place'), 2)		&(('clean', 'rooms'), 3)\\
			(('friendly', 'staff'), 2)		&(('new', 'room'), 3)\\
			(('terrible', 'service'), 2)		&(('next', 'morning'), 3)\\
			(('horrible', 'smell'), 2)		&(('complimentary', 'breakfast'), 3)\\
			(('happy', 'camper'), 2)		&(('free', 'shuttle'), 3)\\
			(('classic', 'wash'), 2)		&(('first', 'night'), 3)\\
			(('synthetic', 'change'), 2)		&(('first', 'day'), 2)\\
		\end{tabular}
	\end{table}
\end{center}
\begin{center}
	\tiny
	\begin{table}[!h]
		\caption{Restaurant reviews}
		\begin{tabular}{c | c | c}
			\textbf{Business 3}			&\textbf{Business 4}				&\textbf{Business 5}	\\\hline
			R4EhR8xhONLFqqI6ZnzNWw		&c1\char`_adyjYG6JEa1PZAXMOBg		&DcfkRb2bS2c8z21WH-aS6A\\\hline
			(('good', 'dumpling'), 8)		&(('south', 'indian'), 14)		&(('carne', 'asada'), 13)\\
			(('good', 'food'), 8)		&(('indian', 'food'), 14)		&(('mexican', 'food'), 5)\\
			(('korean', 'food'), 7)		&(('indian', 'restaurant'), 5)		&(('free', 'chips'), 4)\\
			(('korean', 'dishes'), 7)		&(('other', 'places'), 4)		&(('great', 'place'), 4)\\
			(('steamed', 'dumplings'), 6)		&(('first', 'time'), 4)		&(('authentic', 'food'), 3)\\
			(('chinese', 'food'), 5)		&(('friendly', 'staff'), 4)		&(('red', 'sauce'), 3)\\
			(('chinese', 'dumplings'), 4)		&(('good', 'food'), 4)		&(('mexican', 'restaurants'), 3)\\
			(('chinese', 'cuisine'), 4)		&(('first', 'experience'), 3)		&(('good', 'food'), 3)\\
			(('korean', 'soup'), 4)		&(('high', 'price'), 3)		&(('friendly', 'staff'), 3)\\
			(('great', 'service'), 4)		&(('second', 'time'), 3)		&(('best', 'food'), 3)\\
			(('fried', 'pork'), 4)		&(('indian', 'buffet'), 3)		&(('little', 'flavor'), 2)\\
			(('huge', 'fan'), 3)		&(('great', 'taste'), 2)		&(('toasted', 'bread'), 2)\\
			(('cheap', 'food'), 3)		&(('crispy', 'dosa'), 2)		&(('iced', 'tea'), 2)\\
			(('awesome', 'dumpling'), 3)		&(('last', 'night'), 2)		&(('reasonable', 'price'), 2)\\
			(('fresh', 'noodle'), 3)		&(('indian', 'place'), 2)		&(('many', 'restaurants'), 2)\\
			(('north', 'korean'), 3)		&(('great', 'place'), 2)		&(('many', 'people'), 2)\\
			(('other', 'dishes'), 3)		&(('delicious', 'food'), 2)		&(('good', 'salsa'), 2)\\
			(('fried', 'rice'), 3)		&(('decent', 'reviews'), 2)		&(('great', 'tacos'), 2)\\
			(('hidden', 'gem'), 3)		&(('indian', 'cuisine'), 2)		&(('good', 'taco'), 2)\\
			(('northern', 'chinese'), 3)		&(('tasty', 'food'), 2)		&(('favorite', 'place'), 2)\\
			
		\end{tabular}
	\end{table}
\end{center}
\newpage
\begin{table}[!h]
	\centering
	\tiny
	\caption{Application result}
	\begin{tabular}{c|c}
		\hline
		\multicolumn{2}{c}{Business ID: daqYMX3Y4QR8xl-BUlYBPw}\\
		\multicolumn{2}{c}{ }\\
		\hline
		Good features &'fresh fruit': 5, 'great service': 3, 'great bread': 2,\\
		&'fresh avocado': 2, 'good torta': 2, 'delicious food': 2,\\
		&'great food': 2, 'good food': 2, 'great variety': 1,\\
		&'tasty sandwich': 1, 'tasty pan': 1, 'fine dining': 1,\\
		&'great value': 1, 'friendly place': 1, 'delicious pork': 1,\\
		&'fresh salad': 1, 'excellent staff': 1, 'good value': 1,\\
		&'good sandwich': 1, 'nice tang': 1, 'wonderful quesadilla': 1,\\
		&'great torta/taco': 1, 'wonderful stuff': 1, 'good fun': 1,\\
		&'good variety': 1, 'good quality': 1, 'great atmosphere': 1,\\
		&'good sign': 1, 'good flavor': 1, 'great family': 1,\\
		&'excellent food': 1, 'good nonetheless': 1, 'great place': 1,\\
		&'delicious tomato': 1, 'fresh cream': 1, 'awesome service': 1,\\
		&'excellent breakfast': 1, 'delicious chicken': 1,\\
		&'friendly service': 1, 'fresh avacado': 1, 'great flavor': 1\\
		&\\
		\hline
		Bad features &'service was not great': 1, 'bad one': 1,\\
		&'wrong order': 1, 'are now no good': 1, 'poor job': 1\\
		&\\
		\hline
		
	\end{tabular}
\end{table}
	
	
\end{document}
\endinput
%%
%% End of file `sample-sigchi.tex'.
 
