The results are displayed in Table 5 and Table 6.
\subsection{Approach}
For the extraction of the Noun-Adjectives, we used a dependency-grammar approach. In dependency grammar, adjectives are labeled as 'amod' for sentences like "... bad service ..." while in the case of Adjectives in sentences like "... the service at ... is bad ...", the adjectives are labeled as 'acomp'. With this labels we were able to identify the adjectives. For the nouns we searched the dependency tree of the parent node of the adjective for 
the 'nsubj' relation. All of these relations were being stored in an array. After another check of the POS-Tag of the collected 'nsubj' relations, we were able to write the ADJ-NOUN Tuples to the final array. Before the adding we lemmatized the words so we can have more accurate number of the used words in the final. We decided to do so since our corpus is to small to distinct between different tenses and therefore increasing the accuracy for the summary of the used adjectives and nouns. The lemmatization was done after the Adjective and Noun were identified For the research we looked at the top twenty appearing pairs.
In the results we were able to see some characteritics of the reviewed business. It  was possible for example, to make an assumption about the business of the following review:

After reviewing Business 2 (oLb3-eXUFtCFJl2DuBhcvA)  we were able to assume, that the review describes an hotel. Some other reviews also indicated the sector of the business. 
It was also possible to identify the kind of food which is served in some restaurants, like mexican, vietnamese or chinese for example.
\subsection{Analysis}
After reviewing a few reviews on Business 1, it became obvious, that this business is a car wash. The reviews were mixed, some criticized alot and other reviews were mentioning the good result. \\
Business 2 was also a non restaurant business. The noun-adjective pairs such as (‘front’,’desk’), (‘free’,’wifi’), (‘rental’, ‘car’), (‘complimentary’, ‘breakfast’) suggest that the reviews are about an hotel which is true. We also get a general idea from the pairs that the staff, breakfast as well as the facilities are decent which is also true from reading the reviews. Apart from some complaints here and there in the reviews, the guests generally liked the hotel and its free shuttle service. However, some pairs like (‘light’, ‘sleeper’) is useless as it talks about the guest. Also, (‘next’, ‘day’) and (‘first’, ‘day’) occur frequently which does not give us any info.\\
As for Business 3, most reviews spoke great of the dumplings as well as the food, which are the 2 most frequent noun-adjective pairs. Hence, those are accurate. As from the reviews, it is given that the restaurant has a mix of Korean as well as Northern-Chinese dishes which is also something that can be rightly perceived from the frequent noun-adjective pairs obtained, e.g, (‘korean’, ‘food’), (‘korean’, ‘dishes’), (‘chinese’, ‘cuisine’), etc. Fried pork and fried rice are some of the dishes that are sold in the restaurant that people talked of in the reviews. Most reviews talk of how bad the service is. Apparently, the management and staff is not friendly, and the food is brought in quite slowly. This is in contradiction with one of the noun-adjective pairs that we got ((‘great’), (‘service’)).  Perhaps, the noun-adjective pair could not comprehend the meaning of the sentences where service and great would appear together. These are some sentences quoted from the reviews: “If you are expecting great service, this the wrong place for you”, “but service is not so great”.  Therefore, when extracting noun-adjective pairs for these sentences, we obtain (‘great’), (‘service’), although the expression of the sentence is actually negative.\\
Business 4 represents a South Indian restaurant with a buffet set-up. Hence, the first three most frequent adj-noun pairs correctly summarizes the nature of the restaurant and the kind of food being served in the restaurant. Noun-adj pairs like (‘first’, ‘time’) do not hold much importance in the feedback of the restaurant but (‘second’, ‘time’) can refer to customers wanting to come back which speaks good of the restaurant. Besides, this is quite true because there are quite a few reviews that talks about customers wanting to give it another try. Most reviews also speak good of the staff as given by the (‘friendly’, ‘staff’) pair. Most spoken of food item is the dosa and that’s also given as one of the noun-adjective pairs. On the whole, the frequent noun-adjective pairs obtained give an accurate response of the restaurant.\\
For business 5 we found 4 good reviews and 1 bad review. It seems like busines 5 is a mexican restaurant, which is also indicated by the ('mexican', 'restaurants'), 3)- Pair. We also found the pair ('carne', 'asada'), 13) as "carne asada" in the reviews again.\\
	
	\begin{center}
		\tiny
		\begin{table}[!h]
		\caption{Business reviews}
		\begin{tabular}{c | c}
			\textbf{Business 1}			&\textbf{Business 2}\\ \hline
			2xrpo-LXV9uGIwpvy0dwUw		&oLb3-eXUFtCFJl2DuBhcvA\\\hline
			(('clean', 'car'), 4)		&(('front', 'desk'), 20)\\
			(('other', 'locations'), 3)		&(('free', 'wifi'), 5)\\
			(('great', 'job'), 3)		&(('clean', 'room'), 5)\\
			(('basic', 'wash'), 2)		&(('next', 'day'), 5)\\
			(('terrible', 'wash'), 2)		&(('next', 'door'), 4)\\
			(('poor', 'job'), 2)		&(('light', 'sleeper'), 4)\\
			(('high', 'pressure'), 2)		&(('rental', 'car'), 3)\\
			(('terrible', 'job'), 2)		&(('friendly', 'staff'), 3)\\
			(('helpful', 'guy'), 2)		&(('comfortable', 'bed'), 3)\\
			(('horrible', 'service'), 2)		&(('great', 'breakfast'), 3)\\
			(('bad', 'service'), 2)		&(('hot', 'food'), 3)\\
			(('worth', 'place'), 2)		&(('free', 'breakfast'), 3)\\
			(('different', 'options'), 2)		&(('continental', 'breakfast'), 3)\\
			(('only', 'place'), 2)		&(('clean', 'rooms'), 3)\\
			(('friendly', 'staff'), 2)		&(('new', 'room'), 3)\\
			(('terrible', 'service'), 2)		&(('next', 'morning'), 3)\\
			(('horrible', 'smell'), 2)		&(('complimentary', 'breakfast'), 3)\\
			(('happy', 'camper'), 2)		&(('free', 'shuttle'), 3)\\
			(('classic', 'wash'), 2)		&(('first', 'night'), 3)\\
			(('synthetic', 'change'), 2)		&(('first', 'day'), 2)\\
		\end{tabular}
	\end{table}
	\end{center}
	\begin{center}
		\tiny
		\begin{table}[!h]
		\caption{Restaurant reviews}
		\begin{tabular}{c | c | c}
			\textbf{Business 3}			&\textbf{Business 4}				&\textbf{Business 5}	\\\hline
			R4EhR8xhONLFqqI6ZnzNWw		&c1\char`_adyjYG6JEa1PZAXMOBg		&DcfkRb2bS2c8z21WH-aS6A\\\hline
			(('good', 'dumpling'), 8)		&(('south', 'indian'), 14)		&(('carne', 'asada'), 13)\\
			(('good', 'food'), 8)		&(('indian', 'food'), 14)		&(('mexican', 'food'), 5)\\
			(('korean', 'food'), 7)		&(('indian', 'restaurant'), 5)		&(('free', 'chips'), 4)\\
			(('korean', 'dishes'), 7)		&(('other', 'places'), 4)		&(('great', 'place'), 4)\\
			(('steamed', 'dumplings'), 6)		&(('first', 'time'), 4)		&(('authentic', 'food'), 3)\\
			(('chinese', 'food'), 5)		&(('friendly', 'staff'), 4)		&(('red', 'sauce'), 3)\\
			(('chinese', 'dumplings'), 4)		&(('good', 'food'), 4)		&(('mexican', 'restaurants'), 3)\\
			(('chinese', 'cuisine'), 4)		&(('first', 'experience'), 3)		&(('good', 'food'), 3)\\
			(('korean', 'soup'), 4)		&(('high', 'price'), 3)		&(('friendly', 'staff'), 3)\\
			(('great', 'service'), 4)		&(('second', 'time'), 3)		&(('best', 'food'), 3)\\
			(('fried', 'pork'), 4)		&(('indian', 'buffet'), 3)		&(('little', 'flavor'), 2)\\
			(('huge', 'fan'), 3)		&(('great', 'taste'), 2)		&(('toasted', 'bread'), 2)\\
			(('cheap', 'food'), 3)		&(('crispy', 'dosa'), 2)		&(('iced', 'tea'), 2)\\
			(('awesome', 'dumpling'), 3)		&(('last', 'night'), 2)		&(('reasonable', 'price'), 2)\\
			(('fresh', 'noodle'), 3)		&(('indian', 'place'), 2)		&(('many', 'restaurants'), 2)\\
			(('north', 'korean'), 3)		&(('great', 'place'), 2)		&(('many', 'people'), 2)\\
			(('other', 'dishes'), 3)		&(('delicious', 'food'), 2)		&(('good', 'salsa'), 2)\\
			(('fried', 'rice'), 3)		&(('decent', 'reviews'), 2)		&(('great', 'tacos'), 2)\\
			(('hidden', 'gem'), 3)		&(('indian', 'cuisine'), 2)		&(('good', 'taco'), 2)\\
			(('northern', 'chinese'), 3)		&(('tasty', 'food'), 2)		&(('favorite', 'place'), 2)\\
			
		\end{tabular}
	\end{table}
	\end{center}
	
	\subsection{Conclusion}
	With this summarizer we were able to find very specific characteristics of the business. There are alot of useful adjectives to specific offers of the restaurant (e.g.: (('best', 'buffet'), 4), (('quick', 'service'), 2), (('good', 'food'), 5)). With this results we can get extract the most important pairs of the reviews. But still there are some pairs which are not useful to extract the main information of the review like time information which are not useful without their contextes (e.g. : ('single', 'time'), 2),(('first', 'night'), 3), (('first', 'day'), 2)). We assume that the reason for this is the ambigousness of 
	the POS-Tagging. Neither NLTK nor spacy where able to distinguish between the function of a determiner and an adjective. Since these time words can also be used in other contextes as adjectives, an exclusion of these words would not be reasonable. There were also some inaccurancies apperearing like  (('tomato', 'sauce'), 3) which are not usefull for the reflection.
	We can conclude, that we can extract alot of useful knowledge from the reviews using this Adjective-Noun-Summarizer. 
	