For the extraction of the Noun-Adjectives, we used a dependency-grammar approach. In dependency grammar, adjectives are labeled as 'amod' for sentences like "... bad service ..." while in the case of Adjectives in sentences like "... the service at ... is bad ...", the adjectives are labeled as 'acomp'. With this labels we were able to identify the adjectives. For the nouns we searched the dependency tree of the parent node of the adjective for 
	the 'nsubj' relation. All of these relations were being stored in an array. After another check of the POS-Tag of the collected 'nsubj' relations, we were able to write the ADJ-NOUN Tuples to the final array. Before the adding we lemmatized the words so we can have more accurate number of the used words in the final. We decided to do so since our corpus is to small to distinct between different tenses and therefore increasing the accuracy for the summary of the used adjectives and nouns. The lemmatization was done after the Adjective and Noun were identified For the research we looked at the top twenty appearing pairs.
	In the results we were able to see some characteritics of the reviewed business. It  was possible for example, to make an assumption about the business of the following review:

	After reviewing Business 2 (oLb3-eXUFtCFJl2DuBhcvA)  we were able to assume, that the review describes an hotel. Some other reviews also indicated the sector of the business. 

	After reviewing a few reviews on Business 1 (2xrpo-LXV9uGIwpvy0dwUw), it became obvious, that this business is a car wash. The reviews were mixed, some criticized alot and other reviews were mentioning the good result.  
	It was also possible to identify the kind of food which is served in some restaurants, like mexican, vietnamese or chinese for example.

	
	
	\begin{center}
		\tiny
		\begin{table}[!h]
		\caption{business reviews}
		\begin{tabular}{c c}
			2xrpo-LXV9uGIwpvy0dwUw		&oLb3-eXUFtCFJl2DuBhcvA\\
			(('clean', 'car'), 4)		&(('front', 'desk'), 20)\\
			(('other', 'locations'), 3)		&(('free', 'wifi'), 5)\\
			(('great', 'job'), 3)		&(('clean', 'room'), 5)\\
			(('basic', 'wash'), 2)		&(('next', 'day'), 5)\\
			(('terrible', 'wash'), 2)		&(('next', 'door'), 4)\\
			(('poor', 'job'), 2)		&(('light', 'sleeper'), 4)\\
			(('high', 'pressure'), 2)		&(('rental', 'car'), 3)\\
			(('terrible', 'job'), 2)		&(('friendly', 'staff'), 3)\\
			(('helpful', 'guy'), 2)		&(('comfortable', 'bed'), 3)\\
			(('horrible', 'service'), 2)		&(('great', 'breakfast'), 3)\\
			(('bad', 'service'), 2)		&(('hot', 'food'), 3)\\
			(('worth', 'place'), 2)		&(('free', 'breakfast'), 3)\\
			(('different', 'options'), 2)		&(('continental', 'breakfast'), 3)\\
			(('only', 'place'), 2)		&(('clean', 'rooms'), 3)\\
			(('friendly', 'staff'), 2)		&(('new', 'room'), 3)\\
			(('terrible', 'service'), 2)		&(('next', 'morning'), 3)\\
			(('horrible', 'smell'), 2)		&(('complimentary', 'breakfast'), 3)\\
			(('happy', 'camper'), 2)		&(('free', 'shuttle'), 3)\\
			(('classic', 'wash'), 2)		&(('first', 'night'), 3)\\
			(('synthetic', 'change'), 2)		&(('first', 'day'), 2)\\
		\end{tabular}
	\end{table}
	\end{center}
	\begin{center}
		\tiny
		\begin{table}[!h]
		\caption{business reviews2}
		\begin{tabular}{c c c}
			R4EhR8xhONLFqqI6ZnzNWw		&c1\char`_adyjYG6JEa1PZAXMOBg		&DcfkRb2bS2c8z21WH-aS6A\\
			(('good', 'dumpling'), 8)		&(('south', 'indian'), 14)		&(('carne', 'asada'), 13)\\
			(('good', 'food'), 8)		&(('indian', 'food'), 14)		&(('mexican', 'food'), 5)\\
			(('korean', 'food'), 7)		&(('indian', 'restaurant'), 5)		&(('free', 'chips'), 4)\\
			(('korean', 'dishes'), 7)		&(('other', 'places'), 4)		&(('great', 'place'), 4)\\
			(('steamed', 'dumplings'), 6)		&(('first', 'time'), 4)		&(('authentic', 'food'), 3)\\
			(('chinese', 'food'), 5)		&(('friendly', 'staff'), 4)		&(('red', 'sauce'), 3)\\
			(('chinese', 'dumplings'), 4)		&(('good', 'food'), 4)		&(('mexican', 'restaurants'), 3)\\
			(('chinese', 'cuisine'), 4)		&(('first', 'experience'), 3)		&(('good', 'food'), 3)\\
			(('korean', 'soup'), 4)		&(('high', 'price'), 3)		&(('friendly', 'staff'), 3)\\
			(('great', 'service'), 4)		&(('second', 'time'), 3)		&(('best', 'food'), 3)\\
			(('fried', 'pork'), 4)		&(('indian', 'buffet'), 3)		&(('little', 'flavor'), 2)\\
			(('huge', 'fan'), 3)		&(('great', 'taste'), 2)		&(('toasted', 'bread'), 2)\\
			(('cheap', 'food'), 3)		&(('crispy', 'dosa'), 2)		&(('iced', 'tea'), 2)\\
			(('awesome', 'dumpling'), 3)		&(('last', 'night'), 2)		&(('reasonable', 'price'), 2)\\
			(('fresh', 'noodle'), 3)		&(('indian', 'place'), 2)		&(('many', 'restaurants'), 2)\\
			(('north', 'korean'), 3)		&(('great', 'place'), 2)		&(('many', 'people'), 2)\\
			(('other', 'dishes'), 3)		&(('delicious', 'food'), 2)		&(('good', 'salsa'), 2)\\
			(('fried', 'rice'), 3)		&(('decent', 'reviews'), 2)		&(('great', 'tacos'), 2)\\
			(('hidden', 'gem'), 3)		&(('indian', 'cuisine'), 2)		&(('good', 'taco'), 2)\\
			(('northern', 'chinese'), 3)		&(('tasty', 'food'), 2)		&(('favorite', 'place'), 2)\\
			
		\end{tabular}
	\end{table}
	\end{center}
	
	
	With this summarizer we were able to find very specific characteristics of the business. There are alot of useful adjectives to specific offers of the restaurant (e.g.: (('best', 'buffet'), 4), (('quick', 'service'), 2), (('good', 'food'), 5)). With this results we can get extract the most important pairs of the reviews. But still there are some pairs which are not useful to extract the main information of the review like time information which are not useful without their contextes (e.g. : ('single', 'time'), 2),(('first', 'night'), 3), (('first', 'day'), 2)). We assume that the reason for this is the ambigousness of 
	the POS-Tagging. Neither NLTK nor spacy where able to distinguish between the function of a determiner and an adjective. Since these time words can also be used in other contextes as adjectives, an exclusion of these words would not be reasonable. There were also some inaccurancies apperearing like  (('tomato', 'sauce'), 3) which are not usefull for the reflection.
	We can conclude, that we can extract alot of useful knowledge from the reviews using this Adjective-Noun-Summarizer. 
	