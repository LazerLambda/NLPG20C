The results are displayed in Table 6 and Table 7.
\subsection{Approach}
For the extraction of the Noun-Adjectives, we used a dependency-grammar approach. In dependency grammar, relations with adjectives are labeled as 'amod' for sentences like "... bad service ..." while in the case of Adjectives in sentences like "... the service at ... is bad ...", the adjectives are labeled as 'acomp'. With this two labels we were able to identify the adjectives. In the case of 'amods', the parent node is checked for being a Noun. If there is a POS-tagged 'NOUN' here, the pair is found. For 'acomp' adjectives all child nodes of the parent node were checked. If there is a relation 'nsubj' here, the pair is found. Before adding the pairs, the words are being lemmatized so the result is a more accurate number of the used words. We decided to do so since our corpus is to small to distinct between different tenses. To prevent incorrect POS tagged words from being included in the results, lemmatization takes place after identification. All of these relations were being stored in an array.\\
\begin{Verbatim}[fontsize=\tiny]
# the case that the adjective is a direct amod
if token.dep_ == 'amod' and token.pos_ == "ADJ":
	if token.head.pos_ == 'NOUN':

		# lemmatize the tokens
		if token.head.lemma_ == '-PRON-':
			pair = ( token.lemma_.lower(), e.text.lower() )
		else:
			pair = ( 
				token.lemma_.lower(), 
				token.head.lemma_.lower() 
				)
		pairs.append(pair)

# the case that the adjective is a adj complement
if token.dep_ == 'acomp' and token.pos_ == "ADJ":
	noun = [
		e for e in token.head.children if e.dep_ == "nsubj"
		]
	if noun:
		for e in noun:

		# skip non nouns
		if not e.pos_ == 'NOUN':
		continue

		# lemmatize the tokens
		if e.lemma_ == '-PRON-':
			pair = ( token.lemma_.lower(), e.text.lower() )
		else:
			pair = ( token.lemma_.lower(), e.lemma_.lower() )
			pairs.append(pair)
\end{Verbatim}
\subsection{Analysis}
The pairs created for Business 1 are indicating already the sector of the company. Pairs like ('clean', 'car'), ('basic', 'wash') or ('classic', 'wash') are signs, that we are dealing with the reviews of a car wash here. After reviewing the texts, it became clearer that opinions about this business are mixed. Costumers mentioned bad service, which is also indicated by the determined pairs, like ('poor', 'job'), ('terrible', 'job') or ('bad', 'service'). In contrast, there were also some good reviews about this business, which also came up in the pairs. The most common pair is, for example, ('clean', 'car'), which indicates a good service. In total the results are satisfying, since most of the pairs are providing useful information. The only exception here is ('high', 'pressure') which is too ambiguous.\\
Business 2 was also a non restaurant business. The noun-adjective pairs such as ('front’,’desk’), ('free’,’wifi’), ('rental’, 'car’), ('complimentary’, 'breakfast’) suggest that the reviews are about an hotel which is true. We also get a general idea from the pairs that the staff, breakfast as well as the facilities are decent which is also true from reading the reviews. Apart from some complaints here and there in the reviews, the guests generally liked the hotel and its free shuttle service. However, some pairs like ('light’, 'sleeper’) is useless as it talks about the guest. Also, ('next’, 'day’) and ('first’, 'day’) occur frequently which does not give us any info.\\
As for Business 3, most reviews spoke great of the dumplings as well as the food, which are the 2 most frequent noun-adjective pairs. Hence, those are accurate. As from the reviews, it is given that the restaurant has a mix of Korean as well as Northern-Chinese dishes which is also something that can be rightly perceived from the frequent noun-adjective pairs obtained, e.g, ('korean’, 'food’), ('korean’, 'dishes’), ('chinese’, 'cuisine’), etc. Fried pork and fried rice are some of the dishes that are sold in the restaurant that people talked of in the reviews. Most reviews talk of how bad the service is. Apparently, the management and staff is not friendly, and the food is brought in quite slowly. This is in contradiction with one of the noun-adjective pairs that we got (('great’), ('service’)).  Perhaps, the noun-adjective pair could not comprehend the meaning of the sentences where service and great would appear together. These are some sentences quoted from the reviews: “If you are expecting great service, this the wrong place for you”, “but service is not so great”.  Therefore, when extracting noun-adjective pairs for these sentences, we obtain ('great’), ('service’), although the expression of the sentence is actually negative.\\
Business 4 represents a South Indian restaurant with a buffet set-up. Hence, the first three most frequent adj-noun pairs correctly summarizes the nature of the restaurant and the kind of food being served in the restaurant. Noun-adj pairs like ('first’, 'time’) do not hold much importance in the feedback of the restaurant but ('second’, 'time’) can refer to customers wanting to come back which speaks good of the restaurant. Besides, this is quite true because there are quite a few reviews that talks about customers wanting to give it another try. Most reviews also speak good of the staff as given by the ('friendly’, 'staff’) pair. Most spoken of food item is the dosa and that’s also given as one of the noun-adjective pairs. On the whole, the frequent noun-adjective pairs obtained give an accurate response of the restaurant.\\
For business 5, the determined pairs are providing useful information. A strong point here is that we don't have pairs related to time. The second most often pair is ('mexican', 'food') and the other pair is ('mexican', 'restaurant'), we can easy infer, that this business is a mexican restaurant. The inspection of the reviews proofed this assumption. Most of these reviews were positive towards the business. Determined pairs like ('great', 'place'), ('authentic','food') or ('best','food') emphasizing this observation.In the reiews many little things are mentioned, which is also reflected in the determined pairs. There are examples such as ('carne', 'asada'), ('toasted', 'bread') and ('iced', 'tea'), which indicate such peculiarities. The generated pairs for Business 5 are very unambiguous, so all pairs can be used to retrieve information.
	

	
\subsection{Conclusion}
With this summary we were able to find very specific characteristics of the businesses. There are many useful adjectives for the specific offers of the restaurant (e.g: ((('best', 'buffet'), 4), ((('quick', 'service'), 2), ('good', 'food'), 5))). With these results we can extract the most important pairs of reviews. But there are still some pairs that are not useful for extracting the main information of the reviews, such as time information that is not meaningful without its context( e.g.: (("single", "time"), 2),(("first", "night"), 3), ((("first", "day"), 2)) ). We assume that the reason for this is the problem of ambiguity in POS tagging. Neither NLTK nor Spacy were able to distinguish between the function of a determiner and an adjective. Since these time words can also be used as adjectives in other contexts, an exclusion of these words would not be reasonable. There were also some inaccuracies like (("tomato", "sauce"), 3), which are not useful for reflection.
In conclusion we can state, that we can extract many useful information with this Noun-Adjective Pair Summarizer.